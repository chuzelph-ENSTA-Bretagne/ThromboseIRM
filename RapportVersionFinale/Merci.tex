Je tiens à remercier ici M. Ali Mansour et M. Cexus pour m'avoir aidé à trouver ce PFE en rapport avec les problématiques de la santé. Ce stage fut riche en rencontre très intéressante et ce fut un plaisir de travailler avec les contacts que j'ai pu avoir dont notamment les chercheurs de l'INSERM et les médecins du CHRU de Brest.

Je remercie donc M. Nacim Betrouni et M. Denis Hamad pour leurs conseils et leur accueil au sein de l'INSERM de Lille. Je remercie également les Médecins M. Jean Christophe Gentrix et M. Julien Ognard pour leurs aide et conseils sur l'analyse des images médicales. Je leur dois aussi toute la base de données d'image à laquelle j'ai pu avoir accès.

Ce fut un projet avec peu de rapport avec les thématiques que j'ai pu étudier à l'ENSTA mais je me rends compte à quel point les compétences et expériences que j'ai eu tant pendant mon année militaire que pendant ma formation de trois ans à l'ENSTA m'ont permis d'aboutir à ce résultat. Je tiens donc à remercier tout les professeurs de l'ENSTA pour les cours qu'ils nous ont fait.

J'ai remercie encore une fois M. Ali Mansour pour m'avoir permis de publier, sa patience pour les relectures, le stylo rouge qu'il a sacrifié pour cette tache et la liberté de mouvement qu'il m'a laissé pendant ce PFE afin que je puisse également me former sur d'autres thématiques.
Ce stage m'a permis de me former sur beaucoup de problématiques en rapport avec le milieu médical et rencontrer plusieurs médecins qui m'ont appuyé. Cela m'a demandé de développer un esprit d'initiative afin de voir et résoudre les problèmes qui sont présentés devant moi durant ces 6 mois. J'ai pu approfondir mes connaissances sur des domaines scientifiques comme la classification non supervisée et j'ai pu voir comment il faut analyser des images médicales. Ce stage fut très intéressant et épanouissant.

\medskip

L'objectif que nous nous sommes fixé au départ était de réaliser un programme d'aide au diagnostic pour les pathologies cérébrales. Au final, la chaine de traitement que nous avons mis en place présente plusieurs caractéristiques:

\begin{itemize}
\item elle est très fortement automatisée. L'opérateur qui souhaiterait l'utiliser n'aura qu'à sélectionner une zone d'intérêt suspecte sur l'image IRM et le nombre de tissu qu'il semble voir à l'image.
\item Pour certaines pathologies, l'algorithme développé permet d'isoler la pathologie de la zone présentant des tissus sains.
\item Néanmoins, pour d'autres pathologies, les IRM de perfusion que nous avons utilisé ici ne permettent pas d'aboutir à un résultat.
\end{itemize} 

L'objectif que l'on m'a posé est donc partiellement atteint. Ce résultat est intéressant mais une piste semble prometteuse pour résoudre ce problème . Nous nous sommes ici principalement focalisés sur l'étude d'un seul type d'IRM alors que les médecins se servent de plusieurs modalités d'image IRM pour faire leur diagnostic. L'analyse multimodale semble donc tout indiquée cependant le temps imparti ne m'a pas permis d'implémenter cette solution.
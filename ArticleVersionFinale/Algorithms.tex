
In the litterature, many references about the spectral clustering can be found \cite{von2007tutorial}, \cite{mouysset2010contributions}. However, we will here focus on the Jordan and Weiss spectral clustering algorithm JWSC \cite{von2007tutorial} because is very adapted to the unsupervised classification and it has provided the most satisfying results on the data we are dealing with. At the beginning, our algorithm builds a graph with its similarity matrix in order to evaluate the resemblance of the signal between them. In a second time, we will apply JWSC algorithm in order to cluster the signal thanks to the information extract from the similarity matrix.


\subsection{Definition of the similarity matrix.}

In order to have an algorithm independent from the processed data, we considered only full connected graph because it can estimate the resemblance of each sample with no a priori information. In a full connected graph, all the nodes are connected to each other.

In order to build the similarity matrix $W$, two different options have been studied :

\begin{itemize}

\item First option uses a sparse representation based on building a $L_1$ graph. The main idea consists on rebuild each point $X_i$ with all the other samples, according to a $L_1$ minimisation problem \cite{yan2009semi}.

\item Second option consists on defining $W$ as:
\begin{equation}
w_{ij} = \frac{\|X_i-X_j\|}{2\sigma_i\sigma_j}
\end{equation}
$\sigma_i$ is the coefficient of dispersion in the data around a point $X_i$ , \cite{tartare2014contribution}, \cite{mouysset2010contributions}. During the study, we found that the optimum value of $\sigma_i$ was the distance to the $7^{th}$ closest neighbour of the point $X_i$.

\end{itemize}

During the realization of the tool-chain, the first option was implemented but didn't give satisfaction. However, the second one on the contrary gave better results. The tool-chain use the second method for the construction of the graph and the matrix $W$.


\subsection{Spectral clustering algorithm.}

Many algorithms can be used for the spectral clustering, and some of them have been implemented during this project, \cite{von2007tutorial}, \cite{ng2002spectral},  \cite{shi2000normalized}. We give here details on the Jordan and Weiss algorithm.

To implement the JWSC algorithm, we have firstly to define the Laplacian matrix $L$ of the matrix $W$. The Laplacian is defined by :

\begin{equation}
L = I-D^{-1/2}WD^{-1/2}
\end{equation}

where $D$ is the degree matrix defined by its coefficient $d_{ii} = \sum_j w_{ij}$, a diagonal matrix ad $I$ is the identity matrix.

The processing tool-chain estimates the eigenvector associated to the smallest eigenvalues of $L$, normalizes each of them and realizes a classification using the k-means algorithm. The main steps of the JWSC algorithm are resumed in the algorithm \ref{alg:algorithm1}:

\begin{algorithm}
  \caption{Normalized spectral clustering, Jordan and Weiss }
  
  \textbf{Inputs}% Inputs section
  \begin{algorithmic}[1]
    \STATE Initiate the similarity matrix $W$
    \STATE Define the number of class $N$
  \end{algorithmic}
  \bigskip

  \textbf{Output}% Output section
  \begin{algorithmic}[1]
    \STATE The Cluster table $truth\_table$.
  \end{algorithmic}
  \bigskip
  
  \textbf{Algorithm}
  \begin{algorithmic}[1]
		\STATE Define the symmetric normalized Laplacian $L$ of $W$.
     	\STATE Create a matrix $VectP$ which contains the N eigenvectors associated with the N smallest eigenvalues.
     	\STATE Normalize all the line of $VectP$
     	\STATE Apply the k-means on $VectP$ with $N$ class and put the result in $truth\_table$.	
  \RETURN $truth\_table$
  \end{algorithmic}
  \label{alg:algorithm1}
\end{algorithm}

The main outcome of the algorithm is a truth table clustering all pixels in the ROI.
